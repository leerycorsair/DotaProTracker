\chapter{Исследовательский раздел}
\label{cha:research}

Технические характеристики устройства, на котором выполнялось тестирование:

\begin{itemize}
	\item Операционная система: Windows 10 64-bit \cite{windows}.
	\item Память: 16 GB.
	\item Процессор: AMD Ryzen 5 4600H \cite{amd} @ 3.00 GHz.
\end{itemize}

Тестирование проводилось на ноутбуке при включённом режиме производительности. Во время тестирования ноутбук был нагружен только системными процессами.

Предметом исследования является скорость выполнения запросов к базе данных в зависимости от использования/игнорирования процесса индексации записей в таблице методом бинарного дерева. Следует отметить, что запросы выполнялись на стороне базы данных в многократном количестве, после чего выполнялось усреднение полученных значений.

Индекс был создан по строковому полю \code{signature\_hero} таблицы \code{players}. В качестве тестового запроса был выбран следующий: 

\code{select * from players where signature\_hero = 'Riki';}

В результате эксперимента были полученные значения, представленные в таблице \ref{tabular:times}.


\begin{table}[h!]
	\centering
	\caption{\label{tabular:times}Сравнительный анализ времени выполнения запросов}
\begin{tabular}{|c|c|c|}
	\hline
	\textbf{Количество записей} & \textbf{С индексированием} & \textbf{Без индексирования} \\ \hline
	100 & 0.025 ms & 0.030 ms \\ \hline
	500 & 0.065 ms & 0.145 ms \\ \hline
	2000 & 0.080 ms & 0.475 ms \\ \hline
	5000 & 0.108 ms & 1.040 ms \\ \hline
	10000 & 0.176 ms & 1.976 ms \\ \hline
\end{tabular}
\end{table}

\section*{Вывод}

На основе полученных экспериментальным путем данных можно сделать вывод о том, что использование индексации позволяет существенным образом ускорить выполнение запроса (в ~11,22 раза для таблицы из 10000 записей). Однако следует учитывать тот факт, что на сравнительно небольших выборках данных индексация не дает столько значимого прироста быстродействия системы, а операция индексации требует дополнительной подготовки данных, а также увеличивает объем необходимой памяти для хранения. Так для исследуемой таблицы при 10000 записях потребовалось 88 Кбайт памяти для хранения дополнительных индексов (порядка 10\% от общей памяти, занимаемой таблицей).