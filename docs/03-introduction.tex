\chapter*{ВВЕДЕНИЕ}
\addcontentsline{toc}{chapter}{ВВЕДЕНИЕ}

Сегодня, в век высоких технологий, наравне с общеизвестными спортивными дисциплинами звучит слово <<киберспорт>> и привлекает интерес большого количества людей. Как и в любом виде спорта в киберспорте проводится множество турниров различного уровня во всех регионах с обширным количеством игроков, спонсоров и матчей внутри конкретного соревнования. Одной из самых популярных дисциплин и с огромным количеством зрителей является Dota 2 \cite{twitchtracker}.

Многие люди имеют любимые команды, фаворитов среди игроков и следят за их карьерой. Таким образом, существует потребность в сервисе, предоставляющем данные о турнирах, командах, внутриматчевую или глобальную статистику выступлений и другую сопутствующую информацию.

Целью курсовой работы является проектирование и реализация базы данных, содержащей информацию о киберспортивных матчах и командах в дисциплине Dota 2.

Для достижения указанной выше цели следует выполнить следующие задачи:
	\begin{itemize}
		\item рассмотреть существующие сервисы;
		\item формализовать задание и определить необходимый функционал;
  	\item спроектировать базу данных, описать ее сущности и связи;
		\item выбрать подходящую систему управления базами данных;
        \item реализовать базу данных;
        \item провести сравнительный анализ времени обработки запроса в базе данных с индексацией и без;
		\item реализовать программное обеспечение, которое позволит получить доступ к данным.
	\end{itemize}
